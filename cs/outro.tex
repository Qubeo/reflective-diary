% ===== OUTRO (INDICATORS, BLOOM, TOOLS) =====

\chapter*{Indikátory}
\label{indikatory}
\vspace{-0.5em}
\note{Ukazatel stavu; aspekt výuky, který je možné přímo sledovat.}

\section*{Kvantitativní indikátory}

Práce s časem\\
\note{O kolik minut jsem prodlužoval(a)? Kolik času zůstalo navíc?\\
Jak dlouho jsem měl(a) monolog? Kolik času studenti něco dělali?\\
Kolik času jsem v průměru věnoval(a) jednomu studentovi?}

Interakce se studenty\\
\note{Kolik otázek jsem položil(a)? Kolik z nich bylo uzavřených/otevřených?\\
Kolik otázek jsem dostal(a)? Kolik jsem dokázal(a) zodpovědět?\\
Kolik studentů jsem pochválil(a) (za jakkoliv malý úspěch)?}

Řešení úloh\\
\note{Kolik úloh vyřešili studenti na hodině? Kolik úkolů dostali na doma?\\
Kolik studentů bylo ztraceno (těžký úkol)? Kolik se naopak nudilo?}

Účast\\
\note{Kolik studentů přišlo na výuku? Kolik z nich meškalo?\\
Kolik studentů odešlo během výuky?}

\section*{Kvalitativní indikátory}

Pocity a subjektivní dojmy\\
\note{Jaké pocity jsem měl(a) během výuky?\\
Jací byli studenti? Byli aktivní, poslouchali mě, \dots?\\
Dávám svojí prací studentům dobrý příklad?}

Organizace\\
\note{Věděli studenti, co přesně měli dělat, jak, a proč?\\
Do jaké míry jsem dodržel(a) časový plán?}

Obsah výuky\\
\note{Je moje výuka variabilní (různé typy úloh, pomůcek, \dots)?\\
Odpovídá obsah mojí výuky tomu, co chci naučit?}

\newpage

\chapter*{Bloomova taxonomie}
\label{bloom}
\vspace{-0.5em}
\note{Hierarchie kognitivních vzdělávacích cílů, B.\ Bloom, 1956}
\vspace{-0.3em}

\begin{enumerate}[leftmargin=*]
\item \textbf{Vzpomenout si (\textit{remember})}\\
\note{termíny a fakta, jejich klasifikace a kategorizace}\\
{\small definovat, vyjmenovat, zopakovat, popsat, rozpoznat, přiřadit pojem ke konceptu}

\item \textbf{Pochopit (\textit{understand})}\\
\note{překlad do jiné formy, jednoduchá interpretace, extrapolace}\\
{\small interpretovat, reprezentovat, vysvětlit, parafrázovat, sumarizovat, uvést příklad, ilustrovat na příkladu, porovnat}

\item \textbf{Použít (\textit{apply})}\\
\note{užití postupu ve správné situaci, použití abstrakcí a zobecnění}\\
{\small vykonat, aplikovat, využít, demonstrovat v praxi, implementovat, vyřešit uzavřený a ohraničený problém}

\item \textbf{Analyzovat (\textit{analyze})}\\
\note{rozklad na části, vztahy a interakce mezi částmi}\\
{\small diskutovat, \uv{rozbít} na části, rozlišit, navrhnout/zvolit řešení, organizovat, strukturovat, integrovat, propojit}

\item \textbf{Vyhodnotit (\textit{evaluate})}\\
\note{posouzení na základě stanovených kritérií a standardů}\\
{\small zkontrolovat, detekovat, testovat, posoudit, odborně kritizovat, argumentovat}

\item \textbf{Vytvořit (\textit{create})}\\
\note{vytvoření nového celku, reorganizace do nové struktury}\\
{\small generovat, produkovat, vymyslet, plánovat/projektovat, komponovat, designovat, budovat, kreativně kombinovat prvky}
\end{enumerate}

\note{Hranice mezi úrovněmi nejsou striktně určené. Seznam akcí ti ale pomůže popsat cíle hodiny i znalosti a dovednosti, které by měli studenti získat. Ukáže ti také, jestli je tvá výuka dostatečně variabilní.}

\chapter*{Užitečné nástroje}
\vspace{-0.5em}
\note{Připomenutí nástrojů a konceptů z hodin Praktika vedení cvičení}

\section*{Výuka jako celek}

\begin{itemize}
\item Používání jmen -- \note{jmenovky, tahák, ptaní se}
\item Uspořádání prostoru -- \note{rovnost, tvar usazení, pozice učitele}
\item Verbalizace očekávání -- \note{očekávají obě strany to stejné?}
\item Precedenty -- \note{historie, co se stalo a může se opakovat}
\item Zakázky -- \note{co studenti chtějí dostat?}
\item Hospitace -- \note{jak to dělají jiní? Jak to dělám já?}
\item Žádost o zpětnou vazbu -- \note{je to skutečně tak, jak si myslím?}
\item Bloomova taxonomie -- \note{verbalizace cílů hodiny/výuky}
\item Rubrika -- \note{self-assessment, sledování pokroku}
\end{itemize}

\section*{Struktura hodiny}

\begin{itemize}
\item Situování -- \note{shrnutí kde jsme a kam jdeme}
\item Označení přechodů -- \note{explicitní přechod do dalšího bloku}
\item Veřejný checklist -- \note{viditelná struktura hodiny}
\item Check-in -- \note{otevření, zjištění stavu studentů}
\item Check-out -- \note{uzavření, výzva k reflexi}
\item Tracking -- \note{sledování/uvědomování si vývoje diskuze/hodiny}
\item Parkoviště otázek -- \note{odkládání větších otázek na později}
\end{itemize}

\section*{Zadávání úkolů}

\begin{itemize}
\item Otázky ke skupině -- \note{srozumitelnost, podmínka + akce}
\item Multiple choice questions -- \note{uvěřitelnost možností}
\item Hlasování -- \note{pozitivní, negativní, rozdělování bodů, Kahoot}
\item Jasnost zadání -- \note{začátek, proces, výsledek, hodnocení, trvání}
\item Praktická ukázka -- \note{aby se museli i dívat, nejen poslouchat}
\item Rozcvička -- \note{fyzické/myšlenkové probuzení, opakování}
\item Nastavování prahu -- \note{jak moc snadné je to udělat?}
\item Chunking -- \note{dělení na malý počet stravitelných části}
\item Externí motivace -- \note{body navíc, bonbóny, ...}
\end{itemize}

\section*{Tvorba aktivit}

\begin{itemize}
\item Schody -- \note{postupnost malých, ztěžujících se úkolů}
\item Netradiční zadání -- \note{např.\ programování na mnoho způsobů}
\item Argumentace -- \note{zdůvodnění svého názoru, hodnocení jiných}
\item Figurky -- \note{fyzické objekty a manipulace s nimi}
\item Antiproblém -- \note{řešení opačného problému}
\item Myšlenkový oblak -- \note{sbírání asociací, diskuze, souvislosti}
\item Konceptová mapa -- \note{vizualizace pojmů/znalostí a jejich vztahů}
\end{itemize}

\section*{Uvádění aktivit}

\begin{itemize}
\item Subgrouping -- \note{dělení skupiny na menší části}
\item Harvest/sklizeň -- \note{sdílení názorů, zesílení důležitých signálů}
\item Think-pair-share -- \note{nejprve sami, diskuze v páru, sklízení}
\item Peer review -- \note{studenti si navzájem hodnotí práci}
\item Peer tutoring -- \note{studenti se učí navzájem}
\item Neupozornění na chybu -- \note{studenti si ji najdou sami}
\item Obrácená hodina -- \note{samostudium teorie, společné cvičení}
\end{itemize}

\chapter*{Prostor pro poznámky}
\note{Například nejdůležitější zpětná vazba z hospitací.}

\chapter*{Prostor pro poznámky}
\note{Na co chci myslet, když učím?}
