% ===== INTRODUCTION =====

\chapter*{How does one become a good teacher?}

\vspace*{1em}
Let's pose a dynamic question instead of a static one:

\vspace*{1em}
\textit{\large \enquote{What can you do to improve your teaching skills?}}

\vspace*{1em}
There are three things that seem to be necessary (and probably sufficient) for improvement\punct{.}\footnotemark
\footnotetext{There are also other useful things (such as expert feedback) but those do not scale very well.}
\begin{enumerate}
\item \textbf{Teaching regularly.}\\You need your own taeaching experience, ideally on a regular basis.\item \textbf{Reflecting your teaching.}\\Pay attention to what worked well and what should be changed in the future.
\item \textbf{Observing the teaching of others.}\\Think about what other teachers do, what works for them, what does not and what can you adopt.
\end{enumerate}

This diary will help you reflect your own teaching. It suggests things to pay attention to, questions to ask and aspects to ponder. But bear in mind -- it's not a cookbook for good teaching, it is only a guidebook for your own journey.

To supplement your own teaching reflection, join the conversation with your fellow teachers. Ask around at your institution for a teaching and learning centre or directly engage your colleagues teaching the same course.

\newpage
\section*{How to use the reflective diary?}

First and foremost, use it regularly.

Note down your thoughts both when planning the lecture and after delivering it. There are thirteen spreads (one for each semester week), each suggesting a handful of questions. The diary is quite small -- it's partly to keep your notes short.

After the spreads for individual weeks, there is a rubric for teacher skills, a list of indicators that can be evaluated, a list of some useful teaching tools and some extra space for your own notes and remarks.

\section*{Why to use the reflective diary?}

Regularly using the reflective diary serves mutiple purposes:
\begin{enumerate}[topsep=0pt]
\item It reminds you to reflect your teaching.
\item It provides you with a convenient place to collect notes for the future.
\item It helps you see all the different aspects of teaching.
\item It enables you to track your progress.
\end{enumerate}

\section*{Having any comments or suggestions?}

We highly appreciate suggestions for improvement, notes on your experience or any other comments. Please email Martin at \textit{mukrop@mail.muni.cz.}
