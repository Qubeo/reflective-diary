% ===== REFLECTION ON INDIVIDUAL WEEKS =====

% Margin note
\newcommand{\marginsection}[3]{\marginpar{\raisebox{#2\height}{
\begin{turn}{#1}
%\fontfamily{phv}\selectfont
\bfseries \color{gray} \large
#3\end{turn}}}}

% Timeline for planning the seminar structure
\newcommand{\timeline}{
\vspace*{2.1cm}
\color{gray}
\begin{chronology}[15]{0}{120}{\textwidth}
\end{chronology}
\vspace*{-0.2cm}
\color{black}
}

% Smileys for evaluating teacher's own feelings
\newcommand{\smileys}{
Jak jsem spokojen(a) s hodinou?
\raisebox{-0.3em}{\,\,\Changey[2]{-2}\,\,\Changey[2]{-1}\,\,\Changey[2]{0}\,\,\Changey[2]{1}\,\,\Changey[2]{2}}
}

% Summarizing 2 good and 2 bad points of the week
\newcommand{\goodbadpoints}{
Co se mi povedlo?
\begin{enumerate}
\item
\item
\end{enumerate}

Co mohlo být lépe?
\begin{enumerate}
\item
\item
\end{enumerate}
}

% Common week
\newcommand{\commonweek}{
\chapter{}
\vspace*{-2em}

\marginsection{90}{-1}{před výukou}

Příprava:\hspace{1cm} hodin(y)

Jaký je cíl této hodiny? Proč chci studenty učit tohle?
\vspace*{1cm}

Jak poznám, že se cíle podařilo dosáhnout?

\hspace*{-1cm}
\rule{\rulelength}{0.4pt}

\marginsection{90}{-3}{po výuce}

\smileys

\goodbadpoints

Vlastní otázky:

\newpage
\marginsection{-90}{1}{volné poznámky}
}

\newgeometry{top=0.75cm, bottom=1.5cm, inner=0.75cm, outer=1.2cm,
			heightrounded, marginparwidth=0.5cm, marginparsep=0.3cm}

% ======= WEEK 1 =======

\chapter{}
\vspace*{-2em}

\marginsection{90}{-1.3}{vyplnit před výukou}

Příprava:\hspace{1cm} hodin(y)\\
\note{Příprava materiálů, úkolů, pokynů, ...}

Z jakých bloků se moje hodina skládá?\\
\note{Rozvrhni si 2--6 bloků na časovou osu níže.}

\timeline

Jaké precedenty chci ve výuce nastavit? \\
\note{Oslovování, (ne)formálnost, kladení otázek, začínání včas, ...}
\vspace*{1cm}

\hspace*{-1cm}
\rule{\rulelength}{0.4pt}

\marginsection{90}{-1.5}{vyplnit po výuce}

\smileys
\note{Jaký mám pocit já? Co vidím na studentech?}

\goodbadpoints

\newpage
\marginsection{-90}{1}{volné poznámky}
\vspace*{-2em}
\note{Jaké chtěné/nechtěné precedenty se objevily na hodině?\\
Jak na další hodině udržím ty chtěné a napravím ty nechtěné?\\
Jaká byla atmosféra během vyučování?\\
Rozumí studenti organizaci kurzu a ví, co se od nich očekává?\\
(Další tipy na otázky najdeš na straně \pageref{indikatory}.)}

% ======= WEEK 2 =======

\chapter{}
\vspace*{-2em}

\marginsection{90}{-1.2}{vyplnit před výukou}

Příprava:\hspace{1cm} hodin(y)\\
\note{Příprava materiálů, úkolů, pokynů, ...}

Z jakých bloků se moje hodina skládá?\\
\note{Rozvrhni si 2--6 bloků na časovou osu níže.}

\timeline

Jaké precedenty chci ve výuce udržet? \\
\note{Oslovování, (ne)formálnost, kladení otázek, začátek včas, ...}
\vspace*{1cm}

\hspace*{-1cm}
\rule{\rulelength}{0.4pt}

\marginsection{90}{-1.3}{vyplnit po výuce}

\smileys
\note{Jaký mám pocit já? Co vidím na studentech?}

\goodbadpoints

\newpage
\marginsection{-90}{1}{volné poznámky}
\vspace*{-2em}
\note{Na jaký aspekt výuky se zaměřím při nejbližší hospitaci?\\
Jaký aspekt výuky má sledovat kolega, který navštíví mou hodinu?\\
(Další tipy na otázky najdeš na straně \pageref{indikatory}.)}

% ======= WEEK 3 =======

\chapter{}
\vspace*{-2em}

\marginsection{90}{-1.1}{vyplnit před výukou}

Příprava:\hspace{1cm} hodin(y)

Z jakých bloků se moje hodina skládá?

\timeline

Jaké otázky na hodině položím skupině?
\vspace*{0.8cm}

\hspace*{-1cm}
\rule{\rulelength}{0.4pt}

\marginsection{90}{-1.5}{vyplnit po výuce}

\smileys

\goodbadpoints

Na které otázky skupina reagovala? Na které ne?

\newpage
\marginsection{-90}{1}{volné poznámky}
\vspace*{-2em}
\note{Kolik otázek do publika jsem položil(a)?\\
Kolik otázek položili studenti mně? Je to málo/dost/příliš?\\
(Další tipy na otázky najdeš na straně \pageref{indikatory}.)}

% ======= WEEK 4 =======

\chapter{}
\vspace*{-2em}

\marginsection{90}{-1}{vyplnit před výukou}

Příprava:\hspace{1cm} hodin(y)

Z jakých bloků se moje hodina skládá?

\timeline

\hspace*{-1cm}
\rule{\rulelength}{0.4pt}

\marginsection{90}{-1.6}{vyplnit po výuce}

\smileys

\goodbadpoints

\note{Sem napiš otázku, na kterou si chceš po výuce odpovědět.}\\
Vlastní otázka:

\newpage
\marginsection{-90}{1}{volné poznámky}
\vspace*{-2em}
\note{Jaké znalosti a dovednosti chci naučit? Popiš je ve formátu rubriky, tedy pojmenuj dovednost a popiš škálu (viz str.\ \pageref{rubrika}).\\
Jaký typ učitele chci být? Co bych měl(a) pro to dělat?\\
(Další tipy na otázky najdeš na straně \pageref{indikatory}.)}

% ======= WEEK 5 =======

\chapter{}
\vspace*{-2em}

\marginsection{90}{-1.2}{před výukou}
Příprava:\hspace{1cm} hodin(y)

Jaký je cíl této hodiny? Proč chci studenty učit tohle? \\
\note{Při formulování cílů ti pomůžou slovesa na straně \pageref{bloom}.}
\vspace*{1cm}

Jak poznám, že se cíle podařilo dosáhnout? \\
\note{Existuje test, kterým je možné zjistit, zda jsi cíle dosáhl(a)?}

\hspace*{-1cm}
\rule{\rulelength}{0.4pt}

\marginsection{90}{-2.5}{po výuce}

\smileys

\goodbadpoints

\note{Sem napiš otázku, na kterou si chceš po výuce odpovědět.}\\
Vlastní otázka:

\newpage
\marginsection{-90}{1}{volné poznámky}
\vspace*{-2em}
\note{Podařilo se mi naplnit cíl hodiny?\\
Co je třeba do budoucna na téhle hodině změnit?\\
Jak budu plánovat a strukturovat nejbližší hodinu?\\
(Další tipy na otázky najdeš na straně \pageref{indikatory}.)}

% ======= WEEK 6 =======

\chapter{}
\vspace*{-2em}

\marginsection{90}{-1.2}{před výukou}

Příprava:\hspace{1cm} hodin(y)

Jaký je cíl této hodiny? Proč chci studenty učit tohle? \\
\note{Při formulování cílů ti pomůžou slovesa na straně \pageref{bloom}.}
\vspace*{1cm}

Jak poznám, že se cíle podařilo dosáhnout? \\
\note{Existuje test, kterým je možné zjistit, zda jsi cíle dosáhl(a)?}

\hspace*{-1cm}
\rule{\rulelength}{0.4pt}

\marginsection{90}{-2.5}{po výuce}

\smileys

\goodbadpoints

\note{Sem napiš otázku, na kterou si chceš po výuce odpovědět.}\\
Vlastní otázka:

\newpage
\marginsection{-90}{1}{volné poznámky}
\vspace*{-2em}
\note{Podařilo se mi naplnit cíl hodiny?\\
Co je třeba do budoucna na téhle hodině změnit?\\
Co je problematické na úkolech, které jsem zadal?\\
(Další tipy na otázky najdeš na straně \pageref{indikatory}.)}

% ======= WEEK 7 =======
\commonweek

% ======= WEEK 8 =======
\commonweek

% ======= WEEK 9 =======
\commonweek

% ======= WEEK 10 =======
\commonweek

% ======= WEEK 11 =======
\chapter{}
\vspace*{-2em}

\marginsection{90}{-1}{před výukou}

Příprava:\hspace{1cm} hodin(y)

Jaký je cíl této hodiny? Proč chci studenty učit tohle?
\vspace*{1cm}

Jak poznám, že se cíle podařilo dosáhnout?

\hspace*{-1cm}
\rule{\rulelength}{0.4pt}

\marginsection{90}{-3}{po výuce}

\smileys

\goodbadpoints

Blíží se konec semestru -- dostávají studenti to, co potřebují pro praxi/ke zkoušce?\\
\note{Jestli ne, jaké změny můžu v následujících týdnech provést?}

\newpage
\marginsection{-90}{1}{volné poznámky}

% ======= WEEK 12 =======

\chapter{}
\vspace*{-2em}

\marginsection{90}{-1}{před výukou}

Příprava:\hspace{1cm} hodin(y)

Jaký je cíl této hodiny? Proč chci studenty učit tohle?
\vspace*{1cm}

Jak poznám, že se cíle podařilo dosáhnout?

\hspace*{-1cm}
\rule{\rulelength}{0.4pt}

\marginsection{90}{-3}{po výuce}

\smileys

\goodbadpoints

Jaký je širší kontext toho, co učím?\\
\note{Kde v mojí výuce se dotýkám jiné výuky, jiných oblastí vědění?}

\newpage
\marginsection{-90}{1}{volné poznámky}
\vspace*{-2em}
\note{Jak získám od studentů zpětnou vazbu na ty aspekty mé vyúky, které mě zajímají?\\
Jak to dělají ostatní vyučující?\\
A jak jiné předměty?}

% ======= WEEK 13 =======

\chapter{}
\vspace*{-2em}

\marginsection{90}{-1}{před výukou}

Příprava:\hspace{1cm} hodin(y)

Jaký je cíl této hodiny? Proč chci studenty učit tohle?
\vspace*{1cm}

Jak poznám, že se cíle podařilo dosáhnout?

\hspace*{-1cm}
\rule{\rulelength}{0.4pt}

\marginsection{90}{-3}{po výuce}

\smileys

\goodbadpoints

Co mi dal tento semestr výuky?\\
\note{Co jsem si uvědomil(a)? Co jsem zlepšil(a)?}

\newpage
\marginsection{-90}{1}{volné poznámky}
\vspace*{-2em}
\note{Za co jsou studenti v předmětu hodnoceni?\\
Odpovídá to, co zkouška testuje, tomu, co já učím?\\
Pokud mají studenti zájem se dál samostatně rozvíjet v dané oblasti, jsou na to po tomto kurzu připraveni?}
