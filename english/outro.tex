% ===== OUTRO (INDICATORS, BLOOM, TOOLS) =====

\chapter*{Indicators}
\label{indicators}
\vspace{-0.5em}
\note{Aspects of teaching you can observe directly.}

\section*{Quantitative indicators}

Time management\\
\note{How much longer/shorter was the lesson compared to the plan?\\
How long did I talk? How long were the students actively engaged?\\
What was the average time I spent with a single student?}

Student interaction\\
\note{How many questions have I asked? Were they open- or close-ended?\\
How many questions did I get? How many was I able to answer?\\
How many students did I (verbally) praise/reward for good work?}

Work on exercises\\
\note{How many exercises did students solve during the lecture?\\
How many students got lost in the tasks? How many were bored?}

Attendance\\
\note{How many students attended the lecture? How many were late?\\
How many students left during the lecture?}

\section*{Qualitative indicators}

Feelings and subjective satisfaction\\
\note{What emotions did I have during the lecture?\\
How did the students behave? Were they attentive/active?\\
Am I a good role model for the students?}

Lecture structure\\
\note{Did the students know, what to do, how to do it and why?\\
Did I follow the planned schedule? If not, why?}

Lecture content\\
\note{Is my teaching diverse enough (task types, tools, \dots)?\\
Does my content match the learning goals I want to reach?}

\newpage

\chapter*{Bloom's taxonomy}
\label{bloom}
\vspace{-0.5em}
\note{Hierarchy of cognitive educational objectives, B.\ Bloom, 1956}
\vspace{-0.3em}

\begin{enumerate}[leftmargin=*]
\item \textbf{Remember (\textit{knowledge})}\\
\note{facts and terminology, classification and categorization thereof}\\
{\small define, list, repeat, describe, identify, reproduce, recognize}

\item \textbf{Understand (\textit{comprehension})}\\
\note{reformulation, simple interpretation and extrapolation}\\
{\small rewrite, extend, explain, paraphrase, summarize, give an example, illustrate on an example}

\item \textbf{Apply (\textit{usage})}\\
\note{applying the method in the right situation, abstracting and generalizing}\\
{\small carry out, apply, manipulate, demonstrate, implement, solve a~model problem}

\item \textbf{Analyze (\textit{decomposition})}\\
\note{decomposition into basic blocks, relations and interactions between them}\\
{\small discuss, \enquote{break} into smaller parts, compare and contrast, design/select a solution, deconstruct, interconnect}

\item \textbf{Evaluate (\textit{judgments})}\\
\note{assessment based on set criteria and standards}\\
{\small evaluate, conclude, test, assess, criticize, justify}

\item \textbf{Create (\textit{synthesis})}\\
\note{creating a new product, reorganization into a new structure}\\
{\small generate, modify, rearrange, invent, design, build, compose}
\end{enumerate}

\note{Level boundaries are not strict. The list of actions can help you describe the learning goals/objectives and knowledge/skills the students should gain. It can indicate how diverse your teaching is.}

\chapter*{Useful teaching tools}
\vspace{-0.5em}
\note{A handful of tools and concepts useful for teaching}

\section*{The broader context of teaching}

\begin{itemize}[leftmargin=1.5em]
\item Names and addressing -- \note{first names, nameplates, cheatsheet}
\item Class arrangement -- \note{seating shape, equality, a position of power}
\item Verbalization of expectations -- \note{do both sides expect the same?}
\item Precedents -- \note{common history, things that may repeat}
\item Commissions -- \note{what do the students want?}
\item Visiting other classes -- \note{how do my colleagues teach?}
\item Feedback request -- \note{does my view match the reality?}
\item Bloom's taxonomy -- \note{formulating the goals and objectives}
\item Rubric -- \note{self-assessment, progress tracking}
\end{itemize}

\section*{Lecture structure}

\begin{itemize}[leftmargin=1.5em]
\item Orienting -- \note{summarizing where we are now and where to go}
\item Signposting -- \note{explicitly highlighting block/topic transitions}
\item Public checklist -- \note{visible lecture structure}
\item Check-in -- \note{opening, focusing attention to here and now}
\item Check-out -- \note{conclusion, call for reflection}
\item Tracking -- \note{paying attention to the flow of discussion/lecture}
\item Question parking lot -- \note{openly deferring bigger questions}
\end{itemize}

\section*{Assigning tasks/exercises}

\begin{itemize}[leftmargin=1.5em]
\item Questions for the group -- \note{\enquote{who} + condition + action}
\item Multiple choice questions -- \note{plausibility of choices}
\item Voting -- \note{positive, negative, points distribution, Kahoot}
\item Assignments -- \note{first step, end-product, evaluation, duration}
\item Demonstrations -- \note{engaging multiple senses}
\item Warm-up -- \note{physical/mental waking up, revision}
\item Setting the threshold -- \note{how easy is it to engage in the activity?}
\item Chunking -- \note{grouping items into small \enquote{digestible} units}
\item External motivation -- \note{extra credit, candy, \ldots}
\end{itemize}

\section*{Creating new tasks/exercises}

\begin{itemize}[leftmargin=1.5em]
\item Stairs -- \note{a sequence of small tasks with increasing difficulty}
\item Non-traditional tasks -- \note{e.g., programming in different ways}
\item Reasoning -- \note{justifying one's opinion, considering other views}
\item Tokens -- \note{physical objects and their manipulation}
\item Anti-problem -- \note{solving the reverse problem}
\item Mindmap/cloud -- \note{gathering associations \& relationships}
\item Concept map -- \note{visualization of notions and their relationships}
\end{itemize}

\section*{Class activities}

\begin{itemize}[leftmargin=1.5em]
\item Subgrouping -- \note{splitting the group into smaller parts}
\item Harvesting -- \note{sharing and highlighting significant observations}
\item Think-pair-share -- \note{think for yourself, discuss in pairs, harvest}
\item Peer review -- \note{students evaluate each other's work}
\item Peer tutoring -- \note{students teach each other}
\item Not pointing out mistakes -- \note{let students find them themselves}
\item Flipped class -- \note{self-study theory, exercise together}
\end{itemize}

\chapter*{Your own comments}
\note{E.g., the most relevant feedback from your colleagues.}

\chapter*{Your own comments}
\note{E.g., what should I concentrate on when teaching?}
