% ===== INTRODUCTION =====

\chapter*{How does one become a good teacher?}

\vspace*{1em}
Let's pose a dynamic question instead of a static one:

\vspace*{1em}
\textit{\large \enquote{What can I do to improve my teaching skills?}}

\vspace*{1em}
Three things seem to be necessary (and probably sufficient) for improvement\punct{.}\footnotemark
\footnotetext{There are also other useful things (such as expert feedback), but those do not scale very well.}
\begin{enumerate}
\item \textbf{Teaching regularly.}\\You need your own teaching experience, ideally occurring on a regular basis.
\item \textbf{Reflecting on your teaching.}\\Pay attention to what worked well and what you should change in the future.
\item \textbf{Observing the teaching of others.}\\Think about what other teachers do, what works for them, what does not and what you can adopt.
\end{enumerate}

This diary will help you reflect on your teaching. It suggests things to pay attention to, questions to ask and aspects to ponder. But please bear in mind -- it's not trying to be a cookbook for good teaching, it's only a guidebook for your journey.

To supplement your teaching reflection, join the conversation with your fellow teachers. Ask around at your institution for a teaching and learning center or directly engage your colleagues teaching the same course.

\newpage
\section*{How to use the reflective diary?}

First and foremost, use it regularly.

Note down your thoughts both when planning the lesson and after delivering it. There are fourteen double-page spreads (one for each semester week), each suggesting a handful of questions. The diary has quite a small format -- it's partly to keep your reflections and notes short.

After the page intended for individual weeks, there is a teacher evaluation rubric, a list of indicators to keep track of and evaluate, an index of some useful teaching tools, and an extra space for your notes and remarks.

\section*{Why use the reflective diary?}

Using the reflective diary regularly serves multiple purposes:
\begin{enumerate}[topsep=0pt]
\item It reminds you to reflect on your teaching.
\item It provides you with a convenient place to collect notes for the future.
\item It helps you see all the different aspects of teaching.
\item It enables you to track your progress.
\end{enumerate}

\section*{Have any comments or suggestions?}

We highly appreciate suggestions for improvement, notes on your experience or any other comments. Please email us at \textit{teachinglab@fi.muni.cz}.
